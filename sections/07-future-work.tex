\section{未来的研究}
\label{s:future-work}

在准确而清晰地勾勒\oscoin{}协议的愿景时,我们暂且为一些问题的答案留白。我们将在下面对这些问题进行简短陈述,并准备在未来的工作中逐步解决。

\begin{itemize}
\item 之前在第三章已经强调过信息可信性的问题。我们希望能纳入以预言机为基础的解决方案,并同时鼓励用户们把可疑的项目标记出来。为进一步抑制项目的作恶动机,当检测到项目的不诚实行为时,协议将扣除他们未锁仓的奖励余额作为惩罚。

\item 在第三章中,我们用\osrank{}算法提出了分两步为项目进行排名,第一阶段中选种子集合Υ将作为顶点集,目的是为了摆脱女巫攻击可能带来的困扰。然而,Υ集合如何选出并随轮更新,将作为开放问题以待后续研究。

\item 第二章讨论中提到了我们将继续探索能保障网络安全的proof-of-work共识的候选机制

\item \osrank{}算法中比较重型的工作是对算法各种参数的实验和调整,比如边的权重,阻尼系数,阈值等等,这些实验和调整将持续下去。

\item 当前来看,\osrank{}是衡量软件代码库相对价值的一个很棒的指标,但对于使用专有设置的面向用户的应用和软件,\osrank{}就无法给出有力的变现指标。我们还在寻找能让\osrank{}适用类似场景的方案,这样就可以服务更多的开源项目

\end{itemize}
