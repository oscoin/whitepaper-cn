\section{背景}

\begin{epigraph}{万物开源宣言}
    \noindent 正是在这种情况下我们才更要明白:只有光复开源文化的清朗,才能让一切由开源孵育的可能性峥嵘而出,才能使集体中的分散智慧迸发人性之光,这时候人间居所也将宛如天堂——这个世界的一切皆为了人
\end{epigraph}

\noindent 数字稀缺性的出现,可以在无需信任的第三方参与的情况下,更简单透明地为网络参与者们提供经济激励和动机。比特币的出现和紧跟而来的以太坊引领了众多替代(加密货)币在网络效用和经济机制上的争流热潮。

如果说比特币是依据交易确认和保护网络的行为奖励运营者,那么Zcash和Decred等项目则进一步让开发者和其他的价值创造者们拿到了行为红利。进步虽可观,但我们还缺乏一种措施,为所有无论是直接进行代码贡献还是间接参与开源基础设施建设的人提供可持续的激励。

这只是免费软件面临的众多动机缺乏问题中的一例,也是本文希望着重讨论的问题~\cite{roads and bridges},动机问题将很大程度上影响到整个软件生态。

\subsection{开发者激励和可持续性}
\label{s:incentives}

在我们深溯这项问题前,让我们回顾一下免费开源软件诞生伊始的条件。我们日常使用的软件大部分都是基于免费公开的代码写成。在数字化程度渐深的现代社会,免费软件项目已经成为了社会商品服务衍生与创新的重要支柱。

Github这类软件托管站点和Stack Overflow这类社区的出现让开源成为了广受欢迎的软件工程范式,人人触手可用的高质项目也因此纷纷涌现。这一现象让很多公司的开发周期缩短,市场反应速度加快,而且在开源软件的教育背景下技术人员储备池得以更新,人才招聘因此受益。

当今的很多免费软件项目都是由个人或小团队发起,成立之初往往是为了解决个人,社会抑或是技术相关的问题。在检视他们的动机时,我们常常看到的理由包括个人成就感,为了声誉或为了获取知识,也包括随信仰而来的责任感,以及社群归属感等。

大部分开源软件项目似乎都是基于上述原因开始的,然而那些真正具有里程碑式意义的项目则需要大量的时间与财富资源才能运转下去。这就暴露出一个基本矛盾:现在公众使用的软件设施是基于自愿这一前提开发的。虽有少部分开发者们找到了一些支援工作进行的经济来源,而大部分开发者们还在为响应社区的需求而殚精竭虑,在挤出空闲时间维护代码中捉襟见肘。因此他们不堪重负,放弃项目的结局已屡见不鲜。这里暴露的问题也就是我们想强调的,即,如何发展出一套健康可持续的免费软件运维和融资方式。