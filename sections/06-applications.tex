\section{应用}

\Oscoin{}平台在促进软件社区的新应用诞生方面颇有潜力。相较于以太坊这类应用不可知的虚拟机来说,\Oscoin{}围绕着开源软件的协作流提供了基元设计,让开发者们可以在此之上进行更灵活的组合操作。
当然我们无法穷举出\Oscoin{}所有的应用案例,但下面列出的几个场景无疑从\Oscoin{}平台中受益:

\subsection{治理和集体决策}

我们相信\oscoin{}的智能合约可以帮助开发者们做出集体决策。因为\oscoin{}平台的网络图能够梳理各个项目参与者之间的利益往来并有效激励参与行为。其次,团队内也可以配合其他决策工具使用智能合约来进行争端解决。

\subsection{最小化信任的软件开发过程}
贡献历史被记录上链会催生新的软件开发模式,这种模式几乎无须网络参与者们彼此建立信任。
比起原有的须依赖第三方中介的脆弱的信任模式,\oscoin{}用密码学保证代码提交可信性的方式更能为资源的整合赋能。
\oscoin{}平台的这类特性格外适合对安全需求有要求的软件开发项目,比如航空航天应用和生物医学固件项目,也包括如\oscoin{}这类加密货币项目。

\subsection{激励}
我们认为第三类小有前景的应用方向是为开发者提供激励。在\oscoin{}网络内,这些应用模式可能包括付酬给开发者,为项目之间清算服务协议和依赖关系,或者是利用数字货币及抵押资产创建新的众筹模式,如Patreon。

\subsection{去中心化自治组织}
最后,如果将以上思路与\oscoin{}财政系统(\S\ref{s:treasury})结合起来,那么开源社区就可以逐步衍化成一个源源不断的接受\oscoin{}资助的去中心化自治组织(DAOs)。这样网络内外的所有贡献者都能享受应得的激励,组织内的决策过程将完全透明,开源软件社区运动秉承的真正自由的精神得以永驻。